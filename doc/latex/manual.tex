\documentclass[11pt]{article}

\usepackage{ifpdf}
\usepackage{ifxetex}
\usepackage{color}
\usepackage{amsmath}
\usepackage{amsfonts}
\usepackage{booktabs}
\usepackage{tabularx}
\usepackage{colortbl}
\usepackage{multirow}
\usepackage{xcolor}
\usepackage{subfigure}
\usepackage{dcolumn}
\usepackage{flushend}
\usepackage{capt-of}
\usepackage[T1]{fontenc}
\usepackage{libertine}
\renewcommand*\oldstylenums[1]{{\fontfamily{fxlj}\selectfont #1}}
\usepackage{tikz}
\usetikzlibrary{positioning}
\usepackage{arydshln}
\usepackage{booktabs}
\usepackage{rotating}

\usepackage{graphicx}
\DeclareGraphicsExtensions{.pdf,.eps,.png,.jpg}
\graphicspath{{./figures/}}

\usepackage{url}

% for border on Verbatim environment
\usepackage{fancyvrb}

% for code listings
\usepackage{listings}

% correct bad hyphenation here
\hyphenation{op-tical net-works semi-conduc-tor}

\newcommand{\DATE}{\today} \newcommand{\LTYPE}{latex2e}
\newcommand{\reffig}[1]{Fig.~\ref{fig:#1}}
\newcommand{\reftab}[1]{Table~\ref{tab:#1}}
\newcommand{\refsec}[1]{Section~\ref{sec:#1}}
\newcommand{\refchap}[1]{Chapter~\ref{chap:#1}}
\newcommand{\reflem}[1]{Lemma~\ref{lem:#1}}
\newcommand{\refthm}[1]{Theorem~\ref{thm:#1}}
\newcommand{\refeq}[1]{Equation~(\ref{eq:#1})}
\newcommand{\ceil}[1]{\left\lceil #1 \right\rceil}
\newcommand{\floor}[1]{\lfloor #1 \rfloor}

\definecolor{Navy}{rgb}{0.1,0.1,0.41}
\definecolor{linkcol}   {named}{Navy}
\definecolor{citecol}   {rgb}{.5,0,0}
\definecolor{urlcol}    {rgb}{0,0,1}

\newcommand{\QTWO}{\emph{$Q^2$}}
\newcommand{\Quipu}{\emph{Quipu}}
\newcommand{\QUAD}{{\sc Quad}}
\newcommand{\MAIP}{{\sc Maip}}
\newcommand{\gprof}{\emph{gprof}}
\newcommand{\DWB}{\emph{DWB}}
\newcommand{\DelftWorkbench}{\emph{Delft Workbench}}
\newcommand{\DWARV}{\emph{DWARV}}
\newcommand{\TIMES}{$\times$}
\newcommand{\MCPROF}{\emph{MCProf}}

%-----------------------------------------------------------------------------------------
\begin{document}

% paper title
\title{\MCPROF{}: Memory and Communication Profiler}
\author{Imran Ashraf \\
    Computer Engineering Lab, TU Delft, The Netherlands\\
    I.Ashraf@tudelft.nl
}
\maketitle

%----------------------------------------------------------------------------------
\section{Introduction}
\label{sec:introduction}

\MCPROF{} is a memory and communication profiler. It traces memory reads/writes
and reports memory accesses by various functions in the application as well as
the data-communication between functions. The information is obtained by
performing dynamic binary instrumentation by utilizing Intel Pin \cite{Pin,
Pin_Download} framework.  This manual explains the process of setting up
\MCPROF{} and using it.



%----------------------------------------------------------------------------------
\section{Availability}
\label{sec:availability}

\MCPROF{} can be downloaded from ...


%----------------------------------------------------------------------------------
\section{Required Packages}
\label{sec:reqPackages}

In order to setup and use \MCPROF{} the following two packages are required:

\begin{itemize}

\item Intel Pin DBI framework \cite{Pin_Download} Revision 62732 or higher

\item g++ compiler with support for C++11X

\item  graphviz Dot utility for converting the generated communication graphs
    from DOT to pdf formats

\end{itemize}


%----------------------------------------------------------------------------------
\section{Installation}
\label{sec:installation}

\MCPROF{} uses Makefile to compile the sources. In order to compile \MCPROF{}
from sources on 32-bit / 64-bit Linux, the following steps can be performed.

\begin{itemize}

\item Download Pin and copy and extract it to the directory where you want to
    keep Pin.

\item Define a variable \verb|PIN_ROOT| by running the following commands:

{
\small
\begin{Verbatim}[frame=single]
export PIN_ROOT=/<absolute path to pin>
\end{Verbatim}
}

\item You can also add this line, for instance, to your \textbf{.bashrc} in case
    you are using \textbf{bash} to export the variable automatically on opening
    a terminal.

\item Download \MCPROF{} and copy and extract it to the directory where you want
    to compile it.

\item Go the \MCPROF{} directory and run the following command to compile:

{
\small
\begin{Verbatim}[frame=single]
make
\end{Verbatim}
}

\end{itemize}

If every thing goes fine, you will see a directory \verb|obj-intel64| (or
\verb|obj-ia32| depending upon your architecture). This directory will contain
the executables and object files generated as a result of the compilation. The
important files are:

\begin{itemize}

\item \textbf{mcprof.so} which is the tool. This will be used to profile the
    applications as explained in Section \ref{sec:usage}.

\item executable files of the test applications available in \textbf{tests}
    directory.  These executables can be used as test inputs.

\end{itemize}



%----------------------------------------------------------------------------------
\section{Usage}
\label{sec:usage}

In order to explain the usage of \MCPROF{} we will use the example application
listed in Figure \ref{fig:vectOps}. The complete source-code is available in
\textbf{tests} directory of source package. In this application, $4$ \verb|int|
arrays are created on source lines 23, 24, 25 and 26. These arrays are
initialized in \verb|initVecs| function. The sum and difference of the elements
of these arrays are computed in \verb|sumVecs| and \verb|diffVecs| functions,
respectively.  Finally, these arrays are free on lines 32-35.

\begin{figure} %[ht]
    \centering
%     \lstinputlisting[basicstyle=\scriptsize,language=C,showstringspaces=false,frame=bt]{code/example.c}
    \lstinputlisting
    [
    language=C,
    showstringspaces=false,
    frame=bt,
    numbers=left,
    stepnumber=1,
    basicstyle=\small %\tiny
    ] {code/vectOps.c}
    \caption{Example of an application processing some arrays.}
    \label{fig:vectOps}
\end{figure}


\subsection{Profiling Given Tests}

This example will be compiled during the default compilation of the \MCPROF{}
discussed in Section \ref{sec:installation}. In order to profile this
application by \MCPROF{} you can give the following command:

{
\small
\begin{Verbatim}[frame=single]
make vectOps.test
\end{Verbatim}
}

Similarly, other tests can also be executed by replacing the
\textbf{<vectOps>.test} with the \textbf{<name of the test application>.test} as
given in \textbf{tests} directory. This will generate the output information
depending upon the selected engine. The details of the generated output are
provided in Section \ref{sec:output}.

\subsection{Profiling Your Own Example}

In order to provide an example of how you can compile and profile your own
application, the same \textbf{vectOps} example is provided in directory
\textbf{yourApp}. You can copy this directory to any location you like your
Linux machine. In order to profile this application, a \textbf{makefile} is
provided in this directory containing all the rules to compile and profile this
application. You need to make some changes before profiling this application.

\begin{itemize}

\item Modify the path to Pin directory on line 1 in the makefile.

\item Modify the path to \MCPROF{} directory on line 2 in the makefile.

\end{itemize}

It should be noted that the \MCPROF{} options are provided in \verb|MCPROF\_OPT|
variable in \textbf{makefile}. You can modify these options as required. The
details of these options are available in Section \ref{sec:inputopt}. Once these
modifications are performed, this application can be compiled and profiled by
the following command:

{
\small
\begin{Verbatim}[frame=single]
make mcprof
\end{Verbatim}
}


%----------------------------------------------------------------------------------
\section{\MCPROF{} Input Options}
\label{sec:inputopt}

Follwing are the input options of \MCPROF{}.

\begin{description}

\item [-RecordStack] can be 0 or 1 to tell \MCPROF{} to include stack accesses or not.

\item [-TrackObjects] can be 0 or 1 to tell \MCPROF{} if you want to track objects. If
set 1, the calls to memory allocations functions (malloc, calloc, realloc,
free, new, delete) will be instrumented to track the allocation and deallocation of
objects at run-time in the application. The complete call-path with source
file-name and line-no information will be recorded as well.

\item [-Engine] can be 1,2 or 3. This selects the engine to be used in
    \MCPROF{}.  Based on the selected engine, the desired output is generated as
    below:

    \begin{description}

    \item [Engine 1]    provides the hot-functions and hot-objects (if
        TrackObjects is 1) in the application based on the memory accesses. The
        number of memory reads/writes performed by each function/object is also
        reported.

    \item [Engine 2]    provides the data-communication information between
        functions. If TrackObjects is 1, the communication is also reported
        through objects in the application.

    \item [Engine 3]    reports the memory access information through functions
        and objects per call.

    \end{description}

\end{description}

%----------------------------------------------------------------------------------
\section{\MCPROF{} Generated Output}
\label{sec:output}

\MCPROF{} generates various output files based on the selected engines. An
important file which is generated independent of the selected engine is
\textbf{symbols.out}.  This file contains the information about the
function/object symbols tracked during the application execution. In case of
objects, the allocation address, allocation size, call-path of the allocation is
also reported.

\subsection{Engine 1 Output}

The output of Engine 1 is a text-file \textbf{accesses.out} in the current
directory. This file contains the information about the access performed by
functions/objects in the application. An example output for the \textbf{vectOps}
application is shown below:

{
\scriptsize
\begin{Verbatim}[frame=single]
Function    Total   Reads   Writes  Location
UnknownFtn  286562  232480  54082   :0
initVecs    800     0       800     :0
sumVecs     1200    800     400     :0
diffVecs    1200    800     400     :0
main        74974   70968   4006    :0
Object5     1200    800     400     /<complete path>/vectOps.c:55
Object6     1200    800     400     /<complete path>/vectOps.c:58
Object7     404     4       400     /<complete path>/vectOps.c:61
Object8     404     4       400     /<complete path>/vectOps.c:64
\end{Verbatim}
}

The accesses reported against \textbf{UnknownFtn} are the accesses which cannot
be associated to any function. For instance the accesses before starting the
\textbf{main} function is called.

In case of objects, there source-file and line-no information is also reported in
the last column.

\subsection{Engine 2 Output}

\MCPROF{} records data-communication among functions. This is reported as a
data-communication in DOT format in the file \textbf{communication.dot} in the
current directory. This file can be converted to pdf by the following command:

{
\small
\begin{Verbatim}[frame=single]
dot -Tpdf communication.dot -o communication.pdf
\end{Verbatim}
}

Figure \ref{fig:comm} shows the data-communication graph generated by \MCPROF{}
with TrackObjects as 0. Figure \ref{fig:comm} shows the same graph with
TrackObjects as 1.

\begin{figure}[!h]
\centering
\includegraphics[width=0.85\linewidth]{figures/comm.pdf}

\caption{Data-communication among functions in vectOps application as reported
    by \MCPROF{}. The Grey ovals represent functions. The arcs represent the
    communication with the number on the arc representing the amount of
    data-communication in bytes.}

\label{fig:comm}
\end{figure}

\begin{figure}[!h]
\centering
\includegraphics[width=0.85\linewidth]{figures/commWithObjects.pdf}

\caption{Data-communication among functions in vectOps application as reported
    by \MCPROF{}. The tracked objects and communication through these objects
    is also shown. The Grey ovals represent functions. The white rectangles
    represent objects. The arcs represent the communication with the number on
    the arc representing the amount of data-communication in bytes.}

\label{fig:commWithObjects}
\end{figure}

\subsection{Engine 3 Output}

\MCPROF{} reports per function call accesses in Engine 3 in the text-file
\textbf{percallaccesses.out} as shown below:

{
\scriptsize
\begin{Verbatim}[frame=single]
Printing All Calls

Printing Calls to initVecs
Total Calls : 1
Call No : 0
Call Seq No : 1
Call Stack : UnknownFtn -> main -> initVecs
Writes to Object5 : 400
Writes to Object6 : 400

Printing Calls to sumVecs
Total Calls : 1
Call No : 0
Call Seq No : 2
Call Stack : UnknownFtn -> main -> sumVecs
Reads from Object5 : 400
Reads from Object6 : 400
Writes to Object7 : 400

Printing Calls to diffVecs
Total Calls : 1
Call No : 0
Call Seq No : 3
Call Stack : UnknownFtn -> main -> diffVecs
Reads from UnknownFtn : 3115
Reads from Object5 : 400
Reads from Object6 : 400
Reads from Object7 : 4
Reads from Object8 : 4
Writes to UnknownFtn : 797
Writes to Object8 : 400

Printing Calls to main
Total Calls : 1
Call No : 0
Call Seq No : 0
Call Stack : UnknownFtn -> main
Reads from UnknownFtn : 69368
Writes to UnknownFtn : 3568
\end{Verbatim}
}


%----------------------------------------------------------------------------------
\section{Frequently Encountered Problems}
\label{sec:faq}

This section will cover some of the frequently encountered problems will
setting-up/using \MCPROF{}.

\subsection{Pin Injection Mode Error}

On some systems if Pin (parent) injection mode is not enabled by default then
you see an error as shown below.

{
\small
\begin{Verbatim}[frame=single]
E:Attach to pid 13972 failed. 
E:  The Operating System configuration prevents Pin
E:  from using the default (parent) injection mode.
E:  To resolve, either execute the following (as root):
E:  $ echo 0 > /proc/sys/kernel/yama/ptrace_scope
E:  Or use the "-injection child" option.
E:  For more information, regarding child injection,
E:  see Injection section in the Pin User Manual.
\end{Verbatim}
}

Solution is also suggested in this message, which is to become root and enable
this injection. This can be achieved by running the following two commands:

{
\small
\begin{Verbatim}[frame=single]
sudo -i
echo 0 > /proc/sys/kernel/yama/ptrace_scope
exit
\end{Verbatim}
}



%----------------------------------------------------------------------------------
\section{Contact}
\label{sec:contact}

In case you are interested in contributing to \MCPROF{}, or you have suggestions
for improvements, or you want to report a bug, contact:

\begin{itemize}
\item Imran Ashraf < I.Ashraf@TUDelft.nl >
\end{itemize}

%----------------------------------------------------------------------------------
% references section
\bibliographystyle{unsrt}
\bibliography{bib/references}

% that's all folks
\end{document}
%----------------------------------------------------------------------------------



% An example of an equation \ref{eq:amdhal}.
% \begin{equation}
% \label{eq:amdhal}
% \lim_{p \to \infty} \frac{p}{1-f(p-1)} = \frac{1}{f} = \frac{1}{1-s},
% \end{equation}
%  where $p$ is the speedup factor of the accelerated part, $f$ is the percentual
% contribution of the sequential part, and $s$ is the original percentual
% contribution of the accelerated part.
